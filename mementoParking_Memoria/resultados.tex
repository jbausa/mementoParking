\chapter{Resultados}
\label{chap:resultados}

% Referencias a documentos externos
\externaldocument{metodo}

\drop{E}{n} este capítulo se muestran los resultados obtenidos para la consecución del presente \ac{TFG}. La estructura del capítulo se basa en las iteraciones propias de la metodología elegida y definida en la sección 4 (ver sección \ref{section:desarrollo}). Es necesario llevar a cabo una iteración 0 que servirá para detallar los apartados principales de la gestión del mismo, como los requisitos y el alcance del proyecto, las historias de usuario, la gestión temporal, gestión de usuarios y riesgos y otros apartados necesarios.

Debido al carácter académico del proyecto, los recursos y la gestión de los mismos estarán alejados de los parámetros habituales del mercado laboral, ya que todos los roles necesarios serán desarrollados por dos personas, el autor y el director del proyecto.

\section{Iteración 0}
Esta primera iteración tiene como objetivo definir el alcance del proyecto y su realizar la planificación del mismo, de manera que quede definida una sólida base sobre la que comenzar el desarrollo del mismo. Se estimará el coste temporal y económico, y se abordará la planificación de los recursos que se consideran necesarios para la correcta consecución de los objetivos planteados, tanto técnicos como humanos, y la gestión de los mismos que se pretende llevar a cabo.

	\subsection{Gestión de recursos humanos}
	Tal y como se comentó en la introducción al capítulo, las personas involucradas en el desarrollo del proyecto son el autor y el director del mismo, que coparán todos los roles disponibles, quedando estos de la siguiente manera:
	
	\begin{itemize}[label={$\bullet$},labelindent=\parindent,leftmargin=2cm]
		\item Autor del \ac{TFG}: Equipo de desarrollo, análisis y pruebas del producto.
		\item Director del \ac{TFG}: Propietario, usuario y cliente final del producto.
	\end{itemize}
	
	\subsection{Alcance del proyecto}
		\subsubsection{Descripción del alcance}
		\label{subsubsection:descripcion-alcance}
		El proyecto consistirá en la elaboración de una herramienta accesible a través de una página web, asimismo se proporcionará una aplicación móvil para sistemas operativos Android que facilitará el acceso a la mencionada página.
		La página constará de dos secciones delimitadas por los privilegios necesarios de acceso. 
		
		Una primera sección de acceso libre en la que se encuadran las páginas \textit{Inicio}, \textit{Acerca de}, \textit{Contacto} y aquellas páginas auxiliares necesarias para permitir el registro y autenticación a los usuarios que deseen acceder a la segunda sección.
		Para ganar acceso a la segunda sección será necesario autentificarse mediante un usuario y una contraseña adquiridos mediante un formulario de registro. Esta sección englobará las páginas de gestión de los datos personales del usuario y el acceso a las posiciones geográficas guardadas para este usuario.
		La herramienta permitirá al usuario definir una localización geográfica como posición de aparcamiento de su vehículo utilizando seleccionando la ubicación mediante un mapa interactivo.
		La herramienta mostrará al usuario la localización actual de los elementos guardados previamente mediante un mapa interactivo.
		Los usuarios podrán permitir la compartición de la localización de un elemento con otros usuarios legítimos de la herramienta, llamados \textit{usuarios secundarios}.
		Los \textit{usuarios secundarios} podrán modificar los elementos compartidos bajo las mismas condiciones que el \textit{usuario primario} o usuario propietario del elemento.
		El \textit{usuario primario} podrá revocar los derechos de acceso y edición a los \textit{usuarios secundarios}.
		La herramienta estará diseñada para facilitar el acceso a través de dispositivos móviles mediante los navegadores incorporados.
		Se pondrá a disposición de los usuarios con sistemas operativos Android una aplicación que permitirá el acceso rápido a la herramienta web.
	
		\subsubsection{Criterios de aceptación}
		\label{subsubsection:criterios-aceptacion}
		La herramienta se considerará conforme a criterio siempre que cumpla los puntos detallados a continuación.
		
		\begin{itemize}[label={$\bullet$},labelindent=\parindent,leftmargin=2cm]
			\item La herramienta web debe visualizarse correctamente en los navegadores Midori, Chromium,Chrome y Mozilla Firefox.
			\item La aplicación debe dar acceso a la herramienta web y permitir una correcta visualización de la misma.
			\item La herramienta web debe garantizar el acceso mediante autenticación a los usuarios legítimos, impidiendo el acceso a la sección restringida a los usuarios no identificados.
			\item La herramienta web debe contener mecanismos para el alta de nuevos usuarios y contendrá mecanismos para la recuperación de los datos de entrada por parte de los usuarios legítimos.
			\item La herramienta mostrará un mapa interactivo en el que estarán señalados los elementos guardados por el usuario y permitirá la adición de nuevos elementos o la modificación de los existentes.
			\item La herramienta incorporará mecanismos para permitir la compartición de elementos entre usuarios legítimos y para revocar está compartición.
			\item La herramienta permitirá a los \textit{usuarios secundarios} modificar los elementos compartidos con ellos por los\textit{usuarios primarios}.
			\item La herramienta permitirá a los \textit{usuarios primarios} revocar los derechos de visión y edición de los \textit{usuarios secundarios} a los elementos previamente compartidos.
		\end{itemize}

		\subsubsection{Entregables del proyecto}
		Al finalizar cada una de las iteraciones se entregará al cliente un artefacto en forma de página de web con las características añadidas a la herramienta durante la iteración así como la documentación generada durante esta fase.
		
	Al termino de las iteraciones se entregará la herramienta completa según lo descrito en la descripción del alcance (ver descripción del alcance \ref{subsubsection:descripcion-alcance}) conforme a criterios (ver criterios \ref{subsubsection:criterios-aceptacion}), un manual de usuario de la herramienta, un manual de instalación y una memoria con los documentos generados durante el proceso de desarrollo.
	
		\subsubsection{Suposiciones y restricciones del proyecto}
		Para asegurar el correcto funcionamiento de la herramienta debe tenerse en cuenta los puntos descritos a continuación:
		
		\begin{itemize}[label={$\bullet$},labelindent=\parindent,leftmargin=2cm]
			\item Se asegura el correcto funcionamiento al acceder desde los siguientes navegadores: Midori, Chromium, Chrome y Mozilla Firefox.
			\item La aplicación móvil será compatible con versiones Android 4.1 y posteriores.
		\end{itemize}

	\subsection{Plan del proyecto}
	En función de los requisitos del proyecto, definidos en la sección \textit{Objetivos} \ref{chap:objetivos} y teniendo en cuenta el alcance del proyecto (ver subsección \ref{descripcion-alcance}) y los criterios de aceptación del mismo (ver subsección \ref{subsubsection:criterios-aceptacion})se puede crear la pila de producto, que consistirá en una colección de \textit{historias de usuario} del sistema, su valor de negocio y la estimación temporal para llevarlo a cabo. Tanto el valor de negocio como la estimación han sido llevadas a cabo contando con la ayuda del propietario del producto.
	En la tabla \ref{tab:historia_usuario} podemos observar las historias de usuario retratadas en una pila de producto priorizada en la que se incluye una estimación temporal y el valor de negocio considerado para la misma.
	
	\begin{table}[hp]
	  \centering 
	  \rowcolors{1}{gray!25}{white}
	  \begin{tabular}{p{0.15\linewidth}p{0.5\linewidth}p{0.15\linewidth}p{0.15\linewidth}}
	    \multicolumn{1}{l}{\cellcolor{black!30}\textbf{Identificador}} & 
	 	\multicolumn{1}{c}{\cellcolor{black!30}\textbf{Historia de Usuario}} &
 	 	\multicolumn{1}{c}{\cellcolor{black!30}\textbf{Estimación temporal}} &
 	 	\multicolumn{1}{c}{\cellcolor{black!30}\textbf{Valor de negocio}} 
	 	\\
	    \toprule
		HdU 1	&	Quiero una página web																				&	4 horas	&	Alto	\\
		HdU 2	&	Quiero que los usuarios puedan comunicarse conmigo a través de la página							&	2 horas	&	Bajo	\\
		HdU 3	&	Quiero que aparezcan los datos de la empresa en la página web										&	2 horas	&	Bajo	\\
		HdU 4	&	Quiero que los usuarios se registren para poder acceder 											&	2 horas	&	Alto	\\
		HdU 5	&	Quiero que los usuarios entren mediante una contraseña												&	2 horas	&	Alto	\\
		HdU 6	&	Quiero que los usuarios puedan modificar sus datos													&	4 horas	&	Medio	\\
		HdU 7	& 	Quiero que los usuarios puedan guardar sus posiciones geográficas	mediante un mapa				&	15 horas&	Alto	\\
		HdU 8	&	Quiero que los usuarios puedan ver sus posiciones guardadas en un mapa							&	4 horas	&	Alto	\\
		HdU 9	&	Quiero que los usuarios puedan compartir sus posiciones con otros usuarios						&	15 horas&	Medio	\\
		HdU 10	&	Quiero que los usuarios con acceso a las posiciones guardadas puedan editarlas					&	15 horas&	Bajo	\\
		HdU 11	&	Quiero que los usuarios puedan dejar de compartir sus posiciones guardadas con otros usuarios		&	4 horas	&	Medio	\\
		HdU 12	&	Quiero mostrar el recorrido hasta las posiciones guardadas del usuario en un mapa					&	10 horas&	Bajo	\\
		
		
	    \hline
	  \end{tabular}
	  \caption{Objetivos parciales del \ac{TFG}}
	  \label{tab:historia_usuario}
	\end{table}
	
	\subsection{Gestión temporal del proyecto}
	
	\subsection{Gestión de las comunicaciones}
	La comunicación entre el autor del presente \ac{TFG} y el director del proyecto se realizarán para la información puntual del estado del desarrollo y abordar las posibles dudas generadas durante la realización. Generalmente las reuniones se realizarán semanalmente y se concretarán, preferentemente, mediante correo electrónico.
	Se utilizarán también otro tipo de herramientas para comunicar en todo momento el estado del proyecto y facilitar la comunicación entre los dos actores principales del desarrollo, a saber, el autor y el director. Estas herramientas serán \textit{Trello}, \textit{Github} y \textit{Heroku}.\\
	Trello, tal y como se explico anteriormente (ver subsección \ref{subsubsection:trello}) es un tablero Kanban virtual que permitirá al director estar informado en todo momento de las tareas que el autor está llevando a cabo en cada momento, así como la modificación de la lista de tareas si así lo considerase necesario. Para ello se crea y permite el acceso al director a un nuevo tablero Kanban iniciado para el presente proyecto.
	\todo{Añadir imágenes de Trello con las iteraciones a realizar y alguna realizada}
	
	Github, tal y como se explico anteriormente (ver subsección \ref{subsubsection:github}) es un servidor para repositorios \textit{Git} que se utilizará para el almacenaje en línea tanto del código del proyecto como de la memoria del mismo. Debido a las restricciones existentes en el servicio gratuito, el repositorio es público, por lo que únicamente se procederá a informar al director del proyecto de la dirección donde se encuentra almacenado. Mediante estos actos, se conseguirá que el director tenga acceso completo a toda la documentación generada para la consecución del proyecto pudiendo revisarla cuando así lo que creyera necesario. Esto mismo es aplicable al código fuente del presente proyecto.
	
	Heroku, tal y como se explico anteriormente (ver subsección \ref{subsubsection:heroku}) es el servidor usado para la implantación del presente proyecto, por lo que se utilizará para comprobar los avances en el proyecto mediante la visita a través de un navegador web.
	
	\subsection{Gestión de recursos}
	Tal y como se comentó anteriormente (ver sección %\ref{dispositivos-empleados)
	, para el desarrollo del proyecto se utilizará un equipo con las características detalladas en la tabla \ref{tab:portatil2}. 
	
	\begin{table}[H]
	  \centering 
	  \rowcolors{1}{gray!25}{white}
	  \begin{tabular}{p{0.4\linewidth}p{0.3\linewidth}}
	    \toprule
		Procesador 							& Intel i7-5500U								\\
		Velocidad y Núcleos del Procesador & 2.4 GHz; 2 núcleos 							\\
		Memoria RAM 						& 16 \ac{GB} \ac{DDR}3L \ac{SDRAM} 			\\
		Sistema Operativo					& Elementary \ac{OS} 							\\
	    \hline
	  \end{tabular}
	  \caption{Equipo usado para el desarrollo \ac{TFG}}
	  \label{tab:portatil2}
	\end{table}
	
	Para las pruebas con dispositivos móviles se utilizará un equipo con las características mostradas en la tabla \ref{tab:movil2}.
	\begin{table}[H]
	  \centering 
	  \rowcolors{1}{gray!25}{white}
	  \begin{tabular}{p{0.4\linewidth}p{0.3\linewidth}}
	    \toprule
		Procesador 	& Qualcomm Snapdragon 800		\\
		Velocidad Procesador 	& 2.26 GHz 			\\
		Memoria RAM 			& 2 \ac{GB} 		\\
		Sistema Operativo 		& Android 6.0 		\\
	    \hline
	  \end{tabular}
	  \caption{Equipo usado para las pruebas en dispositivos móviles \ac{TFG}}
	  \label{tab:movil2}
	\end{table}
	
	\subsection{Gestión de riesgos}
	Para la gestión de riesgos del presente proyecto se utiliza una modificación de la lista proporcionada por McConnell en \cite{Mcc97} presentando en la tabla 
	
	%\ref{tab:riesgos} 
	
	los riesgos a los que se podría enfrentar el autor durante el desarrollo, mostrando la probabilidad de que sucedan y el retraso producido en caso de suceso.
	
	\begin{table}[H]
	  \centering 
	  \rowcolors{1}{gray!25}{white}
	  \begin{tabular}{p{0.1\linewidth}p{0.5\linewidth}p{0.15\linewidth}p{0.15\linewidth}}
  	    \multicolumn{1}{l}{\cellcolor{black!30}\textbf{Id}} &
	    \multicolumn{1}{l}{\cellcolor{black!30}\textbf{Riesgo}} & 
	 	\multicolumn{1}{c}{\cellcolor{black!30}\textbf{Probabilidad}} &
 	 	\multicolumn{1}{c}{\cellcolor{black!30}\textbf{Retraso}}
	 	\\	 
	    \toprule
   	    
   	    \multicolumn{1}{r}{\cellcolor{black!30}\textbf{A. }} &
		\multicolumn{3}{l}{\cellcolor{black!30}\textbf{Creación de la planificación}}\\
		A.1. &El esfuerzo es mayor que el estimado.											&	40\%	&	7 horas\\
		A.2. &Un retraso en una tarea produce retrasos en cascada en las tareas dependientes 	&	40\%	&	7 horas\\
		
		\multicolumn{1}{r}{\cellcolor{black!30}\textbf{B. }} &
		\multicolumn{3}{l}{\cellcolor{black!30}\textbf{Organización y gestión}}\\
		B.3. &Afectación por el \textit{Síndrome de la hoja en blanco}							&	20\%	&	10 horas\\
		
		\multicolumn{1}{r}{\cellcolor{black!30}\textbf{C. }} &
		\multicolumn{3}{l}{\cellcolor{black!30}\textbf{Entorno de desarrollo}}\\
		C.4. &Los espacios no están disponibles en el momento necesario						&	60\%	&	12 horas\\
		C.5. &Los espacios están disponibles pero no son adecuados								&	35\%	&	18 horas\\
		C.6. &Los espacios están sobreutilizados, son ruidosos o distraen						&	60\%	&	8 horas\\
		C.7. &La curva de aprendizaje para la nueva herramienta de desarrollo es más larga de lo esperado	&	20\%	&	6 horas\\
		
		\multicolumn{1}{r}{\cellcolor{black!30}\textbf{D. }} &
		\multicolumn{3}{l}{\cellcolor{black!30}\textbf{Usuarios finales}}\\
		&No aplicable&&\\
		
		\multicolumn{1}{r}{\cellcolor{black!30}\textbf{E. }} &
		\multicolumn{3}{l}{\cellcolor{black!30}\textbf{Cliente}}\\
		D.8. &El tiempo de comunicación con el cliente es más lento de lo esperado			&	10\%	&	6 horas\\
		
		\multicolumn{1}{r}{\cellcolor{black!30}\textbf{F. }} &
		\multicolumn{3}{l}{\cellcolor{black!30}\textbf{Personal Contratado}}\\
		&No aplicable&&\\
		
		\multicolumn{1}{r}{\cellcolor{black!30}\textbf{G. }} &
		\multicolumn{3}{l}{\cellcolor{black!30}\textbf{Requisitos}}\\
		F.9. &Los requisitos se han adaptado pero continúan cambiando							&	10\%	& 8 horas\\
		F.10. &Se añaden requisitos extra														&	40\%	& 18 horas\\
		
		\multicolumn{1}{r}{\cellcolor{black!30}\textbf{H. }} &
		\multicolumn{3}{l}{\cellcolor{black!30}\textbf{Producto}}\\ 
		G.11. &El requisito de trabajar con varios sistemas operativos necesita más tiempo del esperado	&	35\%	& 10 horas\\
		G.12. &El trabajo con un entorno software desconocido causa problemas no previstos	&	40\%	& 10 horas\\
		
		\multicolumn{1}{r}{\cellcolor{black!30}\textbf{I. }} &
		\multicolumn{3}{l}{\cellcolor{black!30}\textbf{Fuerzas mayores}}\\
		&No aplicable&&\\	
		
		\multicolumn{1}{r}{\cellcolor{black!30}\textbf{J. }} &
		\multicolumn{3}{l}{\cellcolor{black!30}\textbf{Personal}}\\
		J.13. &La falta de motivación y de moral reduce la productividad						&	5\%		& 4 horas\\
		J.14. &El personal necesita un tiempo extra para acostumbrarse a trabajar con herramientas o entornos nuevos	&	30\%	&	10 horas\\
		J.15. &El personal necesita un tiempo extra para aprender un lenguaje de programación nuevo	&	40\%	&	10 horas\\
		J.16. &El personal trabaja más lento de lo esperado									&	15\%	&	10 horas\\
		
		\multicolumn{1}{r}{\cellcolor{black!30}\textbf{K. }} &
		\multicolumn{3}{l}{\cellcolor{black!30}\textbf{Diseño e implementación}}\\
		K.17. &Un mal diseño implica volver a diseñar e implementar							&	5\%		&	30 horas\\
		K.18. &La utilización de metodologías desconocidas deriva en un periodo extra de formación y tener que volver atrás para corregir los errores iniciales cometidos en la metodología										&	10\%	&	15 horas\\
		
		\multicolumn{1}{r}{\cellcolor{black!30}\textbf{L. }} &
		\multicolumn{3}{l}{\cellcolor{black!30}\textbf{Proceso}}\\	
		&No aplicable&&\\
	    \hline
	  \end{tabular}
	  \caption{Análisis de riesgos \ac{TFG}}
	  \label{tab:riesgos}
	\end{table}
	
	Debido a las características particulares del proyecto, ya que sólo consta de dos actores principales, la mayoría de los riesgos pueden ser omitidos y los que deben fijarse se ven limitados temporalmente a la reacción necesaria por parte del autor.
	
	\subsection{Gestión de costes}
	
	

% Local Variables:
%  coding: utf-8
%  mode: latex
%  mode: flyspell
%  ispell-local-dictionary: "castellano8"
% End:
