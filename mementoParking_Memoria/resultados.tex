\chapter{Resultados}
\label{chapter:resultados}

% Referencias a documentos externos
\externaldocument{metodo}

\drop{E}{n} este capítulo se muestran los resultados obtenidos para la consecución del presente \ac{TFG}. La estructura del capítulo se basa en las iteraciones propias de la metodología elegida y definida en la sección 4 (ver sección \ref{section:desarrollo}). Es necesario llevar a cabo una iteración 0 que servirá para detallar los apartados principales de la gestión del mismo, como los requisitos y el alcance del proyecto, las historias de usuario, la gestión temporal, gestión de usuarios y riesgos y otros apartados necesarios.

Debido al carácter académico del proyecto, los recursos y la gestión de los mismos estarán alejados de los parámetros habituales del mercado laboral, ya que todos los roles necesarios serán desarrollados por dos personas, el autor y el director del proyecto.

\subsection{Iteración 0}
Esta primera iteración tiene como objetivo definir el alcance del proyecto y su realizar la planificación del mismo, de manera que quede definida una sólida base sobre la que comenzar el desarrollo del mismo. Se estimará el coste temporal y económico, y se abordará la planificación de los recursos que se consideran necesarios para la correcta consecución de los objetivos planteados, tanto técnicos como humanos, y la gestión de los mismos que se pretende llevar a cabo.

	\subsection{Gestión de recursos humanos}
	Tal y como se comentó en la introducción al capítulo, las personas involucradas en el desarrollo del proyecto son el autor y el director del mismo, que coparán todos los roles disponibles, quedando estos de la siguiente manera:
	
	\begin{itemize}[label={$\bullet$},labelindent=\parindent,leftmargin=2cm]
		\item Autor del \ac{TFG}: Equipo de desarrollo, análisis y pruebas del producto.
		\item Director del \ac{TFG}: Propietario, usuario y cliente final del producto.
	\end{itemize}
	
	\subsection{Alcance del proyecto}
		\subsubsection{Descripción del alcance}
		\label{descripcion-alcance}
		El proyecto consistirá en la elaboración de una herramienta accesible a través de una página web, asimismo se proporcionará una aplicación móvil para sistemas operativos Android que facilitará el acceso a la mencionada página.
		La página constará de dos secciones delimitadas por los privilegios necesarios de acceso. 
		
		Una primera sección de acceso libre en la que se encuadran las páginas \textit{Inicio}, \textit{Acerca de}, \textit{Contacto} y aquellas páginas auxiliares necesarias para permitir el registro y autenticación a los usuarios que deseen acceder a la segunda sección.
		Para ganar acceso a la segunda sección será necesario autentificarse mediante un usuario y una contraseña adquiridos mediante un formulario de registro. Esta sección englobará las páginas de gestión de los datos personales del usuario y el acceso a las posiciones geográficas guardadas para este usuario.
		La herramienta permitirá al usuario definir una localización geográfica como posición de aparcamiento de su vehículo utilizando seleccionando la ubicación mediante un mapa interactivo.
		La herramienta mostrará al usuario la localización actual de los elementos guardados previamente mediante un mapa interactivo.
		Los usuarios podrán permitir la compartición de la localización de un elemento con otros usuarios legítimos de la herramienta, llamados \textit{usuarios secundarios}.
		Los \textit{usuarios secundarios} podrán modificar los elementos compartidos bajo las mismas condiciones que el \textit{usuario primario} o usuario propietario del elemento.
		El \textit{usuario primario} podrá revocar los derechos de acceso y edición a los \textit{usuarios secundarios}.
		La herramienta estará diseñada para facilitar el acceso a través de dispositivos móviles mediante los navegadores incorporados.
		Se pondrá a disposición de los usuarios con sistemas operativos Android una aplicación que permitirá el acceso rápido a la herramienta web.
	
		\subsubsection{Criterios de aceptación}
		\label{criterios-aceptacion}
		La herramienta se considerará conforme a criterio siempre que cumpla los puntos detallados a continuación.
		
		\begin{itemize}[label={$\bullet$},labelindent=\parindent,leftmargin=2cm]
			\item La herramienta web debe visualizarse correctamente en los navegadores Midori, Chromium,Chrome y Mozilla Firefox.
			\item La aplicación debe dar acceso a la herramienta web y permitir una correcta visualización de la misma.
			\item La herramienta web debe garantizar el acceso mediante autenticación a los usuarios legítimos, impidiendo el acceso a la sección restringida a los usuarios no identificados.
			\item La herramienta web debe contener mecanismos para el alta de nuevos usuarios y contendrá mecanismos para la recuperación de los datos de entrada por parte de los usuarios legítimos.
			\item La herramienta mostrará un mapa interactivo en el que estarán señalados los elementos guardados por el usuario y permitirá la adición de nuevos elementos o la modificación de los existentes.
			\item La herramienta incorporará mecanismos para permitir la compartición de elementos entre usuarios legítimos y para revocar está compartición.
			\item La herramienta permitirá a los \textit{usuarios secundarios} modificar los elementos compartidos con ellos por los\textit{usuarios primarios}.
			\item La herramienta permitirá a los \textit{usuarios primarios} revocar los derechos de visión y edición de los \textit{usuarios secundarios} a los elementos previamente compartidos.
		\end{itemize}

		\subsubsection{Entregables del proyecto}
		Al finalizar cada una de las iteraciones se entregará al cliente un artefacto en forma de página de web con las características añadidas a la herramienta durante la iteración así como la documentación generada durante esta fase.
		
	Al termino de las iteraciones se entregará la herramienta completa según lo descrito en la descripción del alcance (ver descripción del alcance \ref{descripcion-alcance}) conforme a criterios (ver criterios \ref{criterios-aceptacion}), un manual de usuario de la herramienta, un manual de instalación y una memoria con los documentos generados durante el proceso de desarrollo.
	
		\subsubsection{Suposiciones y restricciones del proyecto}
		Para asegurar el correcto funcionamiento de la herramienta debe tenerse en cuenta los puntos descritos a continuación:
		
		\begin{itemize}[label={$\bullet$},labelindent=\parindent,leftmargin=2cm]
			\item Se asegura el correcto funcionamiento al acceder desde los siguientes navegadores: Midori, Chromium, Chrome y Mozilla Firefox.
			\item La aplicación móvil será compatible con versiones Android 4.1 y posteriores.
		\end{itemize}

	\subsection{Plan del proyecto}
	
	\subsection{Gestión temporal del proyecto}
	
	\subsection{Gestión de las comunicaciones}
	
	\subsection{Gestión de recursos}
	
	\subsection{Gestión de riesgos}
	
	\subsection{Gestión de costes}
	

% Local Variables:
%  coding: utf-8
%  mode: latex
%  mode: flyspell
%  ispell-local-dictionary: "castellano8"
% End:
