\chapter{Método de Trabajo}
\label{chap:metodo}

\drop{E}{n} este capítulo se detalla la metodología utilizada para el desarrollo del presente \ac{TFG} así como las tecnologías utilizadas para llevar a término el proyecto.

Para la gestión y el desarrollo del \ac{TFG} se utilizará Kanban y \ac{TDD}, haciendo uso del prototipado evolutivo.

Kanban proporciona un método productivo bien delineado en fases que garantiza que el cambio a la siguiente fase no se producirá hasta haberse completado correctamente la fase actual.

El desarrollo del producto software se realizará mediante \ac{TDD}, ya que es una solución que permite asegurarse del correcto funcionamiento de los componentes antes de permitir el siguiente paso.

Mediante el prototipado evolutivo conseguiremos ir perfilando el producto final mediante un acercamiento por fases terminadas.

Por tanto, el prototipado nos marcará la meta de cada fase, kanban nos indicará los objetivos individuales necesarios para alcanzar el final de cada una de las fases y \ac{TDD} nos permitirá asegurarnos que los componentes desarrollados poseen la calidad exigida por kanban para terminar el desarrollo del objetivo individual.


\section{Dispositivos empleados}
Para el desarrollo del \ac{TFG} se utilizarán los siguientes dispositivos:

	\subsection{Ordenador portátil}
	
	Para el desarrollo del TFG se ha utilizado un ordenador portátil con las características que se detallan en la tabla \ref{tab:portatil}.
	
	\begin{table}[H]
	  \centering 
	  \rowcolors{1}{gray!25}{white}
	  \begin{tabular}{p{0.4\linewidth}p{0.3\linewidth}}
	    \toprule
	    Fabricante 							& Acer 											\\
		Fabricante Procesador 				& Intel 										\\
		Modelo Procesador 					& i7-5500U 										\\
		Velocidad y Núcleos del Procesador & 2.4 GHz; 2 núcleos 							\\
		Memoria RAM 						& 16 \ac{GB} \ac{DDR}3L \ac{SDRAM} 			\\
		Disco Duro Principal y Secundario 	& 250 \ac{GB} \ac{SDD} y 1 \ac{TB} \ac{HDD}	\\
		Sistema Operativo					& Elementary \ac{OS} 							\\
	    \hline
	  \end{tabular}
	  \caption{Ordenador utilizado para el desarrollo \ac{TFG}}
	  \label{tab:portatil}
	\end{table}
	
	\subsection{Teléfono móvil}
	
	Para las pruebas del desarrollo en dispositivos móviles se ha utilizado un teléfono móvil con las características que se detallan en la table \ref{tab:movil}.
	
	\begin{table}[H]
	  \centering 
	  \rowcolors{1}{gray!25}{white}
	  \begin{tabular}{p{0.4\linewidth}p{0.3\linewidth}}
	    \toprule
	    Fabricante 				& LG 				\\
	    Modelo 					& D820 (Nexus 5) 	\\
		Fabricante Procesador 	& Qualcomm 			\\
		Modelo Procesador 		& Snapdragon 800 	\\
		Velocidad Procesador 	& 2.26 GHz 			\\
		Memoria RAM 			& 2 \ac{GB} 		\\
		Sistema Operativo 		& Android 6.0 		\\
		Tamaño Pantalla			& 4.95 pulgadas 	\\
	    \hline
	  \end{tabular}
	  \caption{Dispositivo móvil utilizado para el desarrollo \ac{TFG}}
	  \label{tab:movil}
	\end{table}
	
\section{Kanban}
Kanban es una palabra japonesa que se deriva de \textit{kan} (visual) y \textit{ban} tarjeta. Se desarrolló como parte de una estrategia industrial japonesa para conseguir adecuar la producción a la demanda. Consiste en un sistema de visualización por medio de tarjetas de los recursos en procesos de producción.

Según Scrum Manager, una comunidad profesional para la difusión de Scrum, podríamos definir kanban \cite{Pal15} de la siguiente manera:

\begin{tikzpicture}
	\node[shadowBox] {
	El término kanban aplicado a la gestión ágil de proyectos se refiere a técnicas de representación visual de información para mejorar la eficiencia en la ejecución de las tareas de un proyecto.};
\end{tikzpicture}


	\subsection{Roles}
	
	\subsection{Sprint}
	
	\subsection{Componentes de Kanban}

\section{TDD}

\section{Marco tecnológico}

	\subsection{Herramientas de diseño}
	% Meter los enlaces a mockups
	
	\subsection{Herramientas de gestión del proyecto}
	
	\subsection{Herramientas, tecnologías y frameworks para el desarrollo}
		%Incluir las herramientas de testing,bbdd, sublime
	
	\subsection{Herramientas para la gestión de bases de datos}
	
	\subsection{Herramientas documentales}
	
	\subsection{Herramientas de implantación}

%Métodos loquesea3

% Local Variables:
%  coding: utf-8
%  mode: latex
%  mode: flyspell
%  ispell-local-dictionary: "castellano8"
% End:
