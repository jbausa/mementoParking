\chapter{Conclusiones finales}
\label{chap:conclusiones}
El obetivo de este capítulo consiste en determinar la consecución de los objetivos marcados durante el segundo capítulo (ver \ref{chap:objetivos}), plantear una serie de posibles mejoras a la herramienta y una última sección con la opinión del autor acerca del desarrollo del presente \ac{TFG}.

\section{Análisis de consecución de objetivos}
El objetivo primario que se deseaba alcanzar al término de este proyecto, tal y como se comentó en el segundo capítulo (ver \ref{chap:objetivos}) consistía en:

\begin{tikzpicture}
	\node[shadowBox] {La realización de una página web diseñada para almacenar y recuperar coordenadas geográficas y la modificación de las mismas por parte de varios usuarios.};
\end{tikzpicture}

El objetivo ha sido cumplido satisfactoriamente, logrando crear una herramienta web capaz de recuperar la posición geográfica del usuario, mostrar las ubicaciones guardadas por el mismo y permitir que otros usuarios modificasen las coordenadas de los objetos compartidos. Para marcar como terminado el objetivo principal del desarrollo se marcaron una serie de objetivos parciales, con el fin de asegurar la correcta finalización del trabajo. En la tabla \ref{tab:objetivos2} pueden verse estos objetivos y el momento en que adquirieron la consideración de terminados.

\begin{table}[hp]
  \centering 
  \rowcolors{1}{gray!25}{white}
  \begin{tabular}{p{0.2\linewidth}p{0.5\linewidth}p{0.2\linewidth}}
    \multicolumn{1}{l}{\cellcolor{black!30}\textbf{Id Objetivo}} & 
 	\multicolumn{1}{c}{\cellcolor{black!30}\textbf{Descripción del objetivo}}\\
 	\multicolumn{1}{c}{\cellcolor{black!30}\textbf{Consecución}}\\
    \toprule
    Objetivo 1 & Realizar una página web para el acceso a la herramienta & Sprint 1 \\
	Objetivo 2 & Añadir gestión de usuarios a la página (registro y control de acceso) & Sprint 1 \\
	Objetivo 3 & Permitir el almacenamiento, edición y recuperación de datos a través de la página web & Sprint 3 \\
	Objetivo 4 & Mostrar los datos almacenados mediante la inclusión de un mapa & Sprint 2 \\
	Objetivo 5 & Facilitar a un usuario permitir a otros usuarios la edición de los datos almacenados & Sprint 7\\
	Objetivo 6 & Añadir una opción para mostrar el recorrido desde el punto actual al punto almacenado Sprint 8\\
    \hline
  \end{tabular}
  \caption{Objetivos parciales del \ac{TFG}}
  \label{tab:objetivos2}
\end{table}

\subsection{Objetivo 1}
Se concluye este primer objetivo al término del Sprint 1, mediante la creación de un marco de trabajo en forma de página web que será la que de soporte a la totalidad de la herramienta.

\subsection{Objetivo 2}
Durante el primer Sprint también se da por concluido el segundo objetivo, ya que se dota al producto de una gestión básica de usuarios, permitiendo la creación, edición y eliminación de usuarios en el sistema.

\subsection{Objetivo 3}
La conclusión básica del objetivo se logra al término del tercer Sprint. Se habla de conlusión básica porque en este punto del desarrollo los usuarios únicamente tienen capacidad para almacenar los datos de un elemento. Al término del cuarto Sprint se le dota de la funcionalidad completa, en la que el usuario puede mantener tantos elementos como desee.

\subsection{Objetivo 4}
El objetivo cuatro se completa en el segundo Sprint, con la inclusión de un mapa interactivo en el que se muestran los datos almacenados y se termina de definir en el quinto Sprint, en el que, ante la existencia de varios elementos para mostrar, se añade la funcionalidad de mostrar sólo aquel que el usuario seleccione.

\subsection{Objetivo 5}
Al termino del séptimo Sprint, el objetivo queda satisfecho mediante la incorporación de las herramientas necesarias para conceder permiso a otros usuarios sobre los elementos que el usuario propietario así decida.

\subsection{Objetivo 6}
La consecución de todos los objetivos parciales se da al término del octavo sprint, cuando se muestra en el mapa interactivo una ruta a pie desde la ubicación del usuario hasta el elemento seleccionado.

\section{Propuestas de trabajo}
La herramienta mantiene varios aspectos que pueden ser mejorados. Estos posibles trabajos son:

\begin{itemize}[label={$\bullet$},labelindent=\parindent,leftmargin=2cm]
	\item Añadir autenticación mediante cuentas externas como Google, Facebook o Twitter.
	\item Añadir soporte para redes sociales
	\item Crear una herramienta móvil dedicada para los principales sistemas operativos móviles
	\item Permitir la creación y utilización de elementos de cualquier tipo, no restringiendo el uso únicamente a coches
	\item Permitir un cierto nivel de personalización en la forma de mostrar los datos en el mapa
\end{itemize}

\section{Recapitulación}
Durante el desarrollo del proyecto he conseguido alcanzar un conocimiento profundo de las tecnologías estudiadas, prácticamente todas ellas novedosas, y afianzar en los conceptos estudiados durante la carrera. El estudio de nuevos lenguajes de programación como Ruby on Rails o Javascript, la creación de interfaces adaptables a distintos sistemas, el aprendizaje de multitud de herramientas secundarias tan útiles y usadas en la vida diaria y la adquisición de nuevas competencias en la resolución de problemas de cierto alcance son objetivos nada desdeñables. \\
Durante la preparación de este proyecto he encontrado multitud de problemas a los que he tenido que encontrar respuesta de una u otra manera. La frustración ante fallos con aspecto de irresolubilidad y el aprendizaje de nuevas habilidades que me permitían superarlos, supone una lección extremadamente útil para mi futuro laboral, puesto que las tecnologías cambiarán con el paso del tiempo, pero las formas de enfrentarse a los problemas que surgirán durante su utilización no.

\hfill Juan Bausá Arpón

\hfill En Ciudad Real a 11 de enero de 2016

% Local Variables:
%  coding: utf-8
%  mode: latex
%  mode: flyspell
%  ispell-local-dictionary: "castellano8"
% End: