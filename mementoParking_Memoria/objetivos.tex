\chapter{Objetivos}
\label{chap:objetivos}

\noindent
%Para este capítulo, la normativa indica:
%
%«Concretar y exponer el problema a resolver describiendo el entorno de trabajo,
%la situación y detalladamente qué se pretende obtener. Limitaciones y
%condicionantes a considerar para la resolución del problema (lenguaje de
%construcción, equipo físico, equipo lógico de base y de apoyo, etc.). Si se
%considera necesario, esta sección puede titularse ``Objetivos e hipótesis de
%trabajo''. En este caso, se añadirán las hipótesis de trabajo que el alumno, con
%su TFG, pretende demostrar».

En este capítulo se expone el objetivo principal del TFG, así como los objetivos parciales que se intentarán conseguir con la realización de este trabajo.

\section{Objetivo general}

%El hito final que se pretende lograr, destacando el problema específico que
%resuelve o la funcionalidad que aporta la aplicación o sistema desarrollado.
El objetivo principal de este TFG consiste en el desarrollo de un producto software que permita al usuario almacenar una posición geográfica (típicamente, el lugar de aparcamiento de uno o varios vehículos) y recuperar más tarde esta posición para mostrarla. El usuario podrá utilizar el producto bien desde un navegador web, accediendo a y autenticándose en el servidor, bien a través del dispositivo móvil \ref{fig:Preview_App}. En este último caso, se brinda la opción, una vez recuperada la posición, de mostrar una ruta guiada hasta el lugar de aparcamiento.

%Imagen App (Aproximacion)
\begin{figure}[hbtp]
\centering
\includegraphics[height=60mm, fbox={\fboxrule} 4mm]{images/objetivos/telefono_parking.jpg}
\caption{Aproximación de la aplicación.}
\label{fig:Preview_App}
\end{figure}
%https://dotfirst.io/

\begin{tikzpicture}
	\node[shadowBox] {El objetivo del TFG será la realización de una página web diseñada para almacenar y recuperar coordenadas geográficas y la modificación de las mismas por parte de varios usuarios.};
\end{tikzpicture}

En la figura \ref{fig:arquitectura} se puede ver un diagrama de la arquitectura del sistema que se implementará en el presente TFG.
\begin{figure}[hbtp]
\centering
\includegraphics[scale=0.75, fbox={\fboxrule} 4mm]{images/objetivos/arquitectura.png}
\caption{Arquitectura de la aplicación}
\label{fig:arquitectura}
\end{figure}


\section{Objetivos específicos}

En la tabla \ref{tab:objetivos}

\begin{table}[hp]
  \centering 
  \rowcolors{1}{gray!25}{white}
  \begin{tabular}{p{0.2\linewidth}p{0.7\linewidth}}
    \multicolumn{1}{l}{\cellcolor{black!30}\textbf{Id Objetivo}} & 
 	\multicolumn{1}{c}{\cellcolor{black!30}\textbf{Descripción del objetivo}}\\
    \toprule
    Objetivo 1 & Realizar una página web para el acceso a la herramienta \\
	Objetivo 2 & Añadir gestión de usuarios a la página (registro y control de acceso) \\
	Objetivo 3 & Permitir el almacenamiento, edición y recuperación de datos a través de la página web \\
	Objetivo 4 & Mostrar los datos almacenados mediante la inclusión de mapa \\
	Objetivo 5 & Facilitar a un usuario permitir a otros usuarios la edición de los datos almacenados \\
	Objetivo 6 & Añadir opción para mostrar recorrido desde el punto actual al punto almacenado \\
    \hline
  \end{tabular}
  \caption{Objetivos parciales del TFG}
  \label{tab:objetivos}
\end{table}


\subsection{Objetivo 1}
\emph{Realizar una página web para el acceso a la herramienta.}\\
El comienzo del desarrollo será la implementación de una página web que sirva como marco y base para el resto de objetivos. Al termino de este punto debe existir una página web accesible con todos los elementos típicos que se un usuario espera encontrar, esto es, una página de inicio, contacto, acerca de, y una estructura reconocible y visualmente agradable. También se prestará atención a la accesibilidad desde dispositivos móviles comprobando que la visualización es correcta y no se pierde ni funcionalidad ni estética al cambiar el método de acceso. En la figura \ref{fig:prototipo_Home} se puede observar un prototipo inicial de la página principal de la aplicación.

% IMAGEN: prototipo_Home
\begin{figure}[hbtp]
\centering
\includegraphics[scale=0.5, fbox={\fboxrule} 4mm]{images/objetivos/prototipo_Home.png}
\caption{Prototipo. Home}
\label{fig:prototipo_Home}
\end{figure}

\subsection{Objetivo 2}

\subsection{Objetivo 3}


% Local Variables:
%  coding: utf-8
%  mode: latex
%  mode: flyspell
%  ispell-local-dictionary: "castellano8"
% End:
